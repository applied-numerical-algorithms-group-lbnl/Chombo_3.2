\documentclass[12pt]{article}

\usepackage{amsmath}
\usepackage{amssymb}
\usepackage{epsfig}
\usepackage{subfigure}


\newcommand{\topg}{b} % bedrock topgraphy

\title{Convergence of ``twisty-stream'' problem}
\author{Dan Martin}

\begin{document}


\maketitle

This document presents the convergence of the ``twisty-stream'' example for
the BISICLES code. This is a 2-D doubly-periodic test case in a $160km \times
160 km$ domain. In this
example, the ice has constant thickness $H = 1 km$ on a sloping bed, 
\begin{equation}
\topg = -x cos( \theta )
\end{equation}
where $\theta= 0.1 \deg$. The basal friction is given by 
\begin{equation}
C(x,y) = \beta_o x( 1 + \epsilon + sin(2 \pi  y + m sin(2 \pi x) ) )
\end{equation}
where $\beta_0 = 1000, \epsilon = 0.001$, and $m = 0.25$. A plot of
$C$ is shown in figure \ref{fig:streamFriction}, and a representative plot 
of the velocity magnitude in figure \ref{fig:streamMagVel}.

\begin{figure}
\centering
\epsfig{width=3.5in,figure=streamFriction.eps}
\caption{Basal friction field for ice stream problem}
\label{fig:streamFriction}
\end{figure}


\begin{figure}
\centering
\epsfig{width=3.5in,figure=streamMagVel.eps}
\caption{Magnitude of velocity field for ice stream problem}
\label{fig:streamMagVel}
\end{figure}




\section{Single-level convergence}
We first use Richardson extrapolation to calculate the convergence rate of the
velocity field, which involves pairwise comparison between solutions with grid
spacing $h$ and $2h$. Convergence results are shown in figure
(\ref{fig:RichardsonConverge}); both velocity components exhibit second-order
convergence. 

\begin{figure}
\centering
\subfigure[$x-$velocity]{\epsfig{width=4.0in,figure=xVelRichardson.eps} }
\subfigure[$y-$velocity]{\epsfig{width=4.0in,figure=yVelRichardson.eps}}
\caption{Richardson convergence of velocity}
\label{fig:RichardsonConverge}
\end{figure}

\subsection{L1L2 convergence}
For the L1L2 approximation, we discretize the vertical direction in layers,
which are then vertically integrated. Two velocity fields are computed. The
first is the basal velocity field, which is computed using
vertically-integrated viscosities at cell faces. The second is the flux
velocity field, which is an estimate of the vertically-integrated velocity
field. To evaluate the effect of the number of layers on the solution, we hold
the horizontal resolution fixed at $128 \times 128$ and then increase the
number of vertical layers. We then compare the velocity fields. Figure
\ref{fig:layersConverge} demonstrates that the solution converges at
second-order rates as we increase the number of layers.

\begin{figure}
\centering
\subfigure[$L_1(error)$]{\epsfig{width=2.8in,figure=layers.L1.eps} }
\subfigure[$L_2(error)$]{\epsfig{width=2.8in,figure=layers.L2.eps} }
\subfigure[$L_\infty(error)$]{\epsfig{width=2.8in,figure=layers.max.eps} }

\caption{Convergence of velocity fields as with increased number of vertical layers.}
\label{fig:layersConverge}
\end{figure}

\section{AMR convergence}
To evaluate AMR performance, we compute a uniform-mesh solution on a very fine mesh (in this case $2048 \times 2048$) and then take that as the ``exact'' solution. This will give a slightly different convergence from Richardson extrapolation. However, we can compute the single-grid ``error'' and the AMR ``error'' and then demonstrate that the AMR error is similar to that of the single-level solution with the same effective resolution. This is shown in figures (\ref{fig:xVelAMRConverge}) and (\ref{fig:yVelAMRConverge}), in which we see that for $n_{ref}=2,4,$ and $(2,2)$, we get the same error as the equivalent fine-level solution. 


\begin{figure}
\centering
\subfigure[$L_1(error)$]{\epsfig{width=2.8in,figure=xVel.AMR.L1.eps} }
\subfigure[$L_2(error)$]{\epsfig{width=2.8in,figure=xVel.AMR.L2.eps} }
\subfigure[$L_\infty(error)$]{\epsfig{width=2.8in,figure=xVel.AMR.max.eps} }

\caption{Convergence of $x-$velocity}
\label{fig:xVelAMRConverge}
\end{figure}

\begin{figure}
\centering
\subfigure[$L_1(error)$]{\epsfig{width=2.8in,figure=yVel.AMR.L1.eps} }
\subfigure[$L_2(error)$]{\epsfig{width=2.8in,figure=yVel.AMR.L2.eps} }
\subfigure[$L_\infty(error)$]{\epsfig{width=2.8in,figure=yVel.AMR.max.eps} }

\caption{Convergence of $y-$velocity}
\label{fig:yVelAMRConverge}
\end{figure}



\begin{table}
\centering
\begin{tabular}{|c|c|c|c|c|c|c|c|c|}
\hline
effective &  &  &  &  &  &  &  &  \\
resolution & single-level & rate & $n_{ref}=2$ & rate & $n_{ref} =
4$ & rate & $n_{ref}=(2,2)$ & rate \\
$(1/h)$ &  &  &  &  &  &  &  &  \\
\hline
32  &  &  &  &  &  &  &  &  \\
64  &  &  &  &  &  &  &  &  \\
128  &  &  &  &  &  &  &  &  \\
256  &  &  &  &  &  &  &  &  \\
512  &  &  &  &  &  &  &  &  \\
1024  &  &  &  &  &  &  &  &  \\
\hline
\end{tabular}
\caption{$L_1(x-{\rm vel})$}
\end{table}

\begin{table}
\centering
\begin{tabular}{|c|c|c|c|c|c|c|c|c|}
\hline
effective &  &  &  &  &  &  &  &  \\
resolution & single-level & rate & $n_{ref}=2$ & rate & $n_{ref} =
4$ & rate & $n_{ref}=(2,2)$ & rate \\
$(1/h)$ &  &  &  &  &  &  &  &  \\
\hline
32  &  &  &  &  &  &  &  &  \\
64  &  &  &  &  &  &  &  &  \\
128  &  &  &  &  &  &  &  &  \\
256  &  &  &  &  &  &  &  &  \\
512  &  &  &  &  &  &  &  &  \\
1024  &  &  &  &  &  &  &  &  \\
\hline
\end{tabular}
\caption{$L_2(x-{\rm vel})$}
\end{table}

\begin{table}
\centering
\begin{tabular}{|c|c|c|c|c|c|c|c|c|}
\hline
effective &  &  &  &  &  &  &  &  \\
resolution & single-level & rate & $n_{ref}=2$ & rate & $n_{ref} =
4$ & rate & $n_{ref}=(2,2)$ & rate \\
$(1/h)$ &  &  &  &  &  &  &  &  \\
\hline
32  &  &  &  &  &  &  &  &  \\
64  &  &  &  &  &  &  &  &  \\
128  &  &  &  &  &  &  &  &  \\
256  &  &  &  &  &  &  &  &  \\
512  &  &  &  &  &  &  &  &  \\
1024  &  &  &  &  &  &  &  &  \\
\hline
\end{tabular}
\caption{$L_\infty(x-{\rm vel})$}
\end{table}

\section{Solver performance}
Look at nonlinear solver convergence for the initial velocity solve.

\subsection{Single-level}
Experimenting with various parameters:
\begin{itemize}
\item JFNK vs. Picard (prefer JFNK, but can run into convergence problems if not ``near'' solution)
\item BiCGStab vs. MG (or GMRES) for linear solver. BiCGStab has better convergence properties for highly varying coefficients and is required for JFNK, but may ``hide'' bad behaviors, but pure MG is probably better to look at initially since it's simpler.
\item Arithmetic vs. Harmonic averaging (default had been harmonic, but arithmetic looks better)
\item
\end{itemize}

\end{document}
